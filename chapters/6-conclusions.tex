
\chapter{Conclusions}
\label{ch:conclusion}

This work has introduced sampling-based predictive buffer management, with an openly available implementation in PostgreSQL. This database cache management policy tracks statistics about active queries to estimate future accesses, and uses this information to mimic MIN~\cite{beladyMIN}, the optimal cache replacement algorithm.

Using sampling for PBM provides several advantages over \cite{pbm}, a previous predictive approach that uses a centralized data structure to track access time estimates and make caching decisions. The sampling-based approach is simpler, can be extended and tuned more easily, and generally achieves better results due to an improved strategy for selecting the best eviction candidate based on the statistics.

Sampling-based PBM performs very well on highly sequential workloads, exceeding the performance of both the prior predictive approach and PostgreSQL's existing Clock-sweep strategy by a significant margin. On a mixed analytic workload with both sequential and index scans, extending PBM-sampling to use frequency statistics allows it to perform well, outperforming the prior approach -- which would be more difficult to modify to support new workload types -- and matching the performance of the existing clock-sweep approach.

Over-all, this new approach is ideal for highly sequential workloads while still being competitive for analytic workloads with a mix of sequential and index access.


% TODO and future work?