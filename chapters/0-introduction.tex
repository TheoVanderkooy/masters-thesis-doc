%======================================================================
\chapter{Introduction}
%======================================================================


A crucial aspect of database system performance is managing secondary storage. Many systems are designed for data-sets larger than main memory, and must manage how the available memory is used to maximize performance. Ideally, the data needed to answer queries would always be in-memory at the time it is needed, so that queries are not slowed down waiting for secondary storage. The task of managing what data is kept in memory and what is returned to permanent storage falls on the buffer manager.


% Buffer management is a critical task in database systems.  A database system's buffer manager controls main memory with the task to limit the number of buffers in use so that they fit in the main memory of the machine running the database system.

For most database systems, e.g., PostgreSQL, the buffer manager controls the memory directly.  This means that when requests exceed the available space, the buffer manager has to pick a buffer to evict and replace with the newly requested data.  The critical decision that the buffer manager must make is what block to evict from the buffer (memory) pool when a new block is requested.  For example, PostgreSQL uses the popular Clock algorithm~\cite{pg_buf_readme} to make eviction decisions. % \cite{pg_buf_internals}

The eviction decisions that a buffer manager makes is analogous to cache management/eviction policies that apply generally to all forms of caching, e.g. web caches and CPU caches.  Since reading from secondary storage is significantly slower than reading from memory, it is greatly beneficial to keep as much data in memory as possible.  Since memory is still much more expensive than common forms of secondary storage, systems are typically forced to choose what to keep in memory for faster access within the available limit on the server.  However, the objective is the same as for buffer pool management: how to minimize the number of storage accesses (I/O) by increasing the hit rate of data blocks in memory and improve overall system performance. A key challenge in increasing buffer pool hit rates is to have low running times for the policy that manages the buffer pool through low-latency eviction decisions -- a computationally expensive eviction policy would also increase query latency.

An \textit{optimal} eviction strategy requires knowledge about \textit{future} accesses, making it impossible to implement in a real system. As such, real systems tend to use simple heuristics with low latency computation such as Clock~\cite{gclock} or \gls{lru}.

The optimal policy can be leveraged in the form of \gls{pbm}, where the buffer manager \textit{predicts} future accesses to inform cache eviction decisions and tries to mimic the optimal caching policy. \citet{pbm} use such a strategy, exploiting the structure of sequential database scans to estimate the next access time of data in the cache. Their approach uses a priority-queue based strategy and was originally implemented in a closed source system called Actian Vector. This approach is described in more detail in Chapter~\ref{sec:pbm-pq_summary}.

\textbf{Contributions}: This thesis proposes an alternate approach to predictive buffer management using sampling that is simpler, more flexible and extensible, and generally uses more up-to-date estimates. The sampling-based approach is shown to perform better than the prior priority-queue based approach for some workloads. Moreover, the extensible design of the sampling-based approach allows for expanding the set of workloads on which predictive buffer management can be used. An open-source implementation in PostgreSQL is available of both the sampling-based approach and the prior approach \cite{pbm}, and the details of the implementations are discussed.

The rest of this thesis is organized as follows. Chapter~\ref{ch:background} includes backgroun on aspects of buffer management and caching as well as related work, \Cref{sec:PBM-sampling,ch:extending-sampling} present the sampling-based technique proposed in this thesis, Chapter~\ref{sec:pbm-pq_postgres} describes the PostgreSQL implementation of the technique, Chapter~\ref{ch:evaluation} presents performance evaluation, and Chapter~\ref{ch:conclusion} concludes the thesis.